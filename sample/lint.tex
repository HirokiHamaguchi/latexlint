\documentclass[a4paper, 10pt]{article}

\usepackage{amsfonts}
\usepackage{amsmath}
\usepackage{amsthm}
\usepackage{ascmac}
\usepackage{cleveref}
\usepackage{float}
\usepackage{graphicx}
\usepackage{mathtools}
\usepackage{siunitx}
\usepackage{xcolor}

\definecolor{cA}{HTML}{0072BD}
\definecolor{cB}{HTML}{EDB120}
\definecolor{cC}{HTML}{77AC30}
\definecolor{cD}{HTML}{D95319}

\newcommand{\tA}[1]{\textcolor{cA}{#1}}
\newcommand{\tB}[1]{\textcolor{cB}{#1}}
\newcommand{\tC}[1]{\textcolor{cC}{#1}}
\newcommand{\tD}[1]{\textcolor{cD}{#1}}

\newtheorem{theorem}{Theorem}
\crefname{theorem}{Thm.}{Thms.}

\setlength{\parindent}{0pt}

\begin{document}

\title{Sample document}
\author{hari64boli64}
\date{\today}
\maketitle

\section{LLAlignAnd}

\begin{table}[H]
	\centering
	\begin{tabular}{lll}
		\&=              &
		$\begin{aligned}
				 a & = b \\
				 c & = d
			 \end{aligned}$ &
		\tA{ok}            \\[0.3cm]
		=\&              &
		$\begin{aligned}
				 a = & b \\
				 c = & d
			 \end{aligned}$ &
		\tD{ng}            \\[0.3cm]
		=\{\}\&          &
		$\begin{aligned}
				 a = {} & b \\
				 c = {} & d
			 \end{aligned}$ &
		\tA{ok}
	\end{tabular}
\end{table}

\section{LLAlignSingleLine}

\begin{itembox}{\large Long line before display (same result)}
	\begin{minipage}[t]{.5\columnwidth}
		Lorem ipsum.
		\begin{equation*}
			f(x) = ax^2 + bx + c
		\end{equation*}
		This is an \tA{equation} environment.
	\end{minipage}%
	\begin{minipage}[t]{.5\columnwidth}
		Lorem ipsum.
		\begin{align*}
			f(x) & = ax^2 + bx + c
		\end{align*}
		This is an \tD{align} environment.
	\end{minipage}
\end{itembox}

\vskip\baselineskip

\begin{itembox}{\large Short line before display (different result)}
	\begin{minipage}[t]{.5\columnwidth}
		Lrm:
		\begin{equation*}
			f(x) = ax^2 + bx + c
		\end{equation*}
		This is an \tA{equation} environment.
	\end{minipage}%
	\begin{minipage}[t]{.5\columnwidth}
		Lrm:
		\begin{align*}
			f(x) & = ax^2 + bx + c
		\end{align*}
		This is an \tD{align} environment.
	\end{minipage}
\end{itembox}

Single-line alignat environment is also detected.
\begin{alignat*}{1}
	f(x) & = ax^2 + bx + c
\end{alignat*}

Multi-line alignat environment is not detected.
\begin{alignat*}{2}
	f(x) & = ax^2 + bx + c \\
	g(x) & = dx^2 + ex + f
\end{alignat*}

% Comment in the following lines should be ignored.
% \begin{align}
% \begin{align}
% \begin{align}

\section{LLColonEqq}

\begin{table}[H]
	\centering
	\begin{tabular}{lll}
		x := y                   & $x := y$        & \tD{ng} \\
		x $\backslash$coloneqq y & $x \coloneqq y$ & \tA{ok} \\
		x ::= y                  & $x ::= y$       & \tD{ng} \\
		x $\backslash$Coloneqq y & $x \Coloneqq y$ & \tA{ok}
	\end{tabular}
\end{table}

\section{LLColonForMapping}

\begin{table}[H]
	\centering
	\begin{tabular}{lll}
		\text{A :                 B}                                                               & $A : B$                                 & \tA{ok} \\
		\text{A $\backslash$colon B}                                                               & $A\colon B$                             & \tD{ng} \\
		\text{f(x) :                 $\backslash$mathbb\{R\}$\backslash$to$\backslash$mathbb\{R\}} & $f(x) : \mathbb{R} \to \mathbb{R}$      & \tD{ng} \\
		\text{f(x) $\backslash$colon $\backslash$mathbb\{R\}$\backslash$to$\backslash$mathbb\{R\}} & $f(x) \colon \mathbb{R} \to \mathbb{R}$ & \tA{ok}
	\end{tabular}
\end{table}

\begin{itembox}{We detect all of : in the following}
	Here are examples of colons we detect.
	\begin{itemize}
		\item $X:Y \to Z$,
		\item \( X: Y \mapsto Z. \),
		\item $X : \mathbb{R}^{n^2 + 2n + 1}  \rightarrow \mathbb{R}$
	\end{itemize}
	and
	\begin{equation} \label{eq:sample}
		X:
		(Y \text{ at new line in tex file})
		\to
		(Z \text{ at new line in tex file}).
	\end{equation}
\end{itembox}

\vskip\baselineskip

\begin{itembox}{We do NOT detect any of : in the following}
	Here are examples of `:' we do not detect.
	\begin{itemize}
		\item $X\colon Y \to Z$, the correct use of colon.
		\item $A:B:C = 1:2:3$, the colon for ratio.
		\item $A:B = 1:2$ and $\alpha \to \beta$, separated by dollar sign.
		\item $f: (\text{some very very very very very long long long long words}) \to \mathbb{R}$, the false negative.
	\end{itemize}
\end{itembox}

\section{LLCref}

\begin{theorem}
	\label{thm:sample}
	This is a sample theorem.
\end{theorem}

Use \cref{thm:sample} with cref instead of Theorem~\ref{thm:sample} with ref to avoid mistakes.

\section{LLDoubleQuotation}

Use ``XXX'' instead of “XXX” or "XXX".

\section{LLENDash}

\begin{itemize}
	\item Erdos-Renyi (random graph, Erd\H{o}s--R\'enyi)
	\item Einstein-Podolsky-Rosen (quantum physics, Einstein--Podolsky--Rosen)
	\item Fruchterman-Reingold (graph drawing, Fruchterman--Reingold)
	\item Gauss-Legendre (numerical integration, Gauss--Legendre)
	\item Gibbs-Helmholtz (thermodynamics, Gibbs--Helmholtz)
	\item Karush-Kuhn-Tucker (optimization, Karush--Kuhn--Tucker)
\end{itemize}

Exception: Fritz-John (optimization, name of a person)

False Positive: Wrong-Example

\section{LLEqnarray}

We should not use eqnarray. It has some spacing issues.

\begin{eqnarray}
	x & = & y \\
	a & = & b
\end{eqnarray}

\section{LLLlGg}

\begin{table}[h]
	\centering
	\begin{tabular}{lll}
		n $\backslash$ll m & $n \ll m$ & \tA{ok} \\
		n $<<$ m           & $n<<m$    & \tD{ng} \\
	\end{tabular}
\end{table}

I like human $<<<$ cat $<<<<<<<<<<<<<<<<$ dog.

\section{LLRefEq}

To refer to the equation, use \eqref{eq:sample} with eqref instead of (\ref{eq:sample}) with ref.

You can avoid the mistakes of forgetting to add parentheses.

\section{LLSharp}

\begin{table}[H]
	\centering
	\begin{tabular}{lll}
		$\backslash\#$    & $\#A$      & \tA{ok} \\
		$\backslash$sharp & $\sharp A$ & \tD{ng}
	\end{tabular}
\end{table}

\section{LLNonASCII}

The following line contains non-ASCII characters.

	{ }!"#$%&'()*+,-./

\vskip\baselineskip

日本語の文章は、upLaTeXでフツウに書けます。

(You can write Japanese sentences as usual with upLaTeX.)

\section{LLSI}

Example: 10 KB, $3.5\mathrm{MiB}$, \SI{500}{\giga\byte}.

\newcommand{\EB}{Some Awesome Command.This is not ExaByte.}

\EB.

This 1EB is one ExaByte.

\section{LLT}

\begin{equation*}
	X^T \quad X^\top \quad X^{\mathsf{T}}
\end{equation*}

\section{LLTitle}

\subsection{The quick brown fox jumps over the lazy dog}

\subsubsection{
	This Is a Correct Title
}

\subparagraph{SubParagraph: Test With Ref~\ref{thm:sample}}

\section{LLUserDefined}

You can define your own rule, such as prohibiting the use of a f\^{}a.

\begin{equation*}
	f^a(X) \quad f^{\mathrm{a}}(X)
\end{equation*}

\end{document}
