\documentclass[a4paper, 10pt]{article}

\usepackage{amsmath}
\usepackage{amsfonts}
\usepackage{amssymb}
\usepackage{mathtools}
\usepackage{graphicx}
\usepackage{physics}
\usepackage{cleveref}

\begin{document}

\title{Sample Document}
\author{Hiroki Hamaguchi}
\date{\today}
\maketitle

\section{Rename Command}

\begin{equation}\label{eq:1}
	a = b
\end{equation}

\begin{equation*}
	c = d
\end{equation*}

\begin{align}
	e & = \begin{dcases}
		      f & \text{if } g = h \\
		      i & \text{if } j = k
	      \end{dcases} \\
	l & = m \notag
\end{align}

\begin{figure*}[h] % comment
	\centering
	\includegraphics[width=0.5\columnwidth]{../images/sample.png}
	\caption{Sample Figure}\label{fig:1}
\end{figure*}

Sample figure is shown in Figure~\ref{fig:1}.

\section{Ask Wolfram Alpha}

Select a line. Then, press \texttt{Ctrl + Shift + P} and type \texttt{askWolframAlpha}.
\begin{gather*}
	\sum_{n=1}^{\infty} \frac{1}{n^2} \\
	\int_{-\infty}^{\infty} e^{-x^2} \dd x \\
	\left(1+\left(1+\left(1+x\right)^2\right)^2\right)
\end{gather*}

The Wolfram Alpha page with adjusted expressions will open as in Figure~\ref{fig:2}.
\begin{figure}[htbp]
	\centering
	\includegraphics[width=\columnwidth]{../images/askWolframAlpha3.png}
	\caption{Wolfram Alpha}\label{fig:2}
\end{figure}

\section{
  Exception Handling
 }

Second-Order Optimization.
This hyphen is intentional.
We will ignore it if exception is registered.

\section{Go to Label Definition (F12)}

This feature allows you to jump to the corresponding \texttt{\textbackslash{}label} definition
when you press F12 on \texttt{\textbackslash{}ref}, \texttt{\textbackslash{}cref}, or \texttt{\textbackslash{}Cref} commands.

% This is a comment: \label{eq:commented} should be ignored
\begin{equation}\label{prob:example1}
	\min_x \|Ax - b\|_2^2
\end{equation}

\begin{equation}\label{eq:example2}
	E = mc^2
\end{equation}

Here are some examples:
\begin{itemize}
	\item Press F12 on \texttt{\textbackslash{}ref\{prob:example1\}} to jump: Prob~\ref{prob:example1}
	\item Press F12 on \texttt{\textbackslash{}cref\{eq:example2\}} to jump: \cref{eq:example2}
	\item Press F12 on \texttt{\textbackslash{}Cref\{fig:1\}} to jump to Figure 1: \Cref{fig:1}
\end{itemize}

% Another commented label: \label{eq:another-comment}
The feature ignores labels inside comments and jumps to the first non-commented occurrence.

\end{document}
