\documentclass[a4paper, 10pt]{article}

\usepackage{amsmath}
\usepackage{amsthm}
\usepackage{amsfonts}
\usepackage{ascmac}
\usepackage{siunitx}
\usepackage{cleveref}
\usepackage{mathtools}
\usepackage{graphicx}

\usepackage{xcolor}
\definecolor{cA}{HTML}{0072BD}
\definecolor{cB}{HTML}{EDB120}
\definecolor{cC}{HTML}{77AC30}
\definecolor{cD}{HTML}{D95319}

\newcommand{\tA}[1]{\textcolor{cA}{#1}}
\newcommand{\tB}[1]{\textcolor{cB}{#1}}
\newcommand{\tC}[1]{\textcolor{cC}{#1}}
\newcommand{\tD}[1]{\textcolor{cD}{#1}}

\newtheorem{theorem}{Theorem}
\crefname{theorem}{Thm.}{Thms.}

\begin{document}

\title{Sample document}
\author{hari64boli64}
\date{\today}
\maketitle

\section{LLAlignAnd}

\begin{table}[h]
    \centering
    \begin{tabular}{lll}
        \&=              &
        $\begin{aligned}
                 a & = b \\
                 c & = d
             \end{aligned}$ &
        \tA{ok}            \\[0.3cm]
        =\&              &
        $\begin{aligned}
                 a = & b \\
                 c = & d
             \end{aligned}$ &
        \tD{ng}            \\[0.3cm]
        =\{\}\&          &
        $\begin{aligned}
                 a = {} & b \\
                 c = {} & d
             \end{aligned}$ &
        \tA{ok}
    \end{tabular}
\end{table}

\section{LLColonEqq}

\begin{table}[h]
    \centering
    \begin{tabular}{lll}
        x := y                   & $x := y$        & \tD{ng} \\
        x $\backslash$coloneqq y & $x \coloneqq y$ & \tA{ok} \\
        x ::= y                  & $x ::= y$       & \tD{ng} \\
        x $\backslash$Coloneqq y & $x \Coloneqq y$ & \tA{ok}
    \end{tabular}
\end{table}

\section{LLColonForMapping}

\begin{table}[h]
    \centering
    \begin{tabular}{lll}
        \text{A :                 B}                                                               & $A : B$                                 & \tA{ok} \\
        \text{A $\backslash$colon B}                                                               & $A\colon B$                             & \tD{ng} \\
        \text{f(x) :                 $\backslash$mathbb\{R\}$\backslash$to$\backslash$mathbb\{R\}} & $f(x) : \mathbb{R} \to \mathbb{R}$      & \tD{ng} \\
        \text{f(x) $\backslash$colon $\backslash$mathbb\{R\}$\backslash$to$\backslash$mathbb\{R\}} & $f(x) \colon \mathbb{R} \to \mathbb{R}$ & \tA{ok}
    \end{tabular}
\end{table}

\begin{itembox}{We detect all of : in the following.}
    Here is the example of colon we detect.
    \begin{itemize}
        \item $X:Y \to Z$,
        \item \( X: Y \mapsto Z. \),
        \item $X : \mathbb{R}^{n^2 + 2n + 1}  \rightarrow \mathbb{R}$
    \end{itemize}
    and
    \begin{equation} \label{eq:sample}
        X:
        (Y \text{ at new line in tex file})
        \to
        (Z \text{ at new line in tex file}).
    \end{equation}
\end{itembox}

\begin{itembox}{We do NOT detect any of : in the following.}
    Here is the example of `:' undetected:
    \begin{itemize}
        \item $X\colon Y \to Z$, the correct use of colon.
        \item $A:B:C = 1:2:3$, the colon for ratio.
        \item $A:B = 1:2$ and $\alpha \to \beta$, separated by dollar sign.
        \item $f: (\text{some very very very very very long long long long words}) \to \mathbb{R}$, the false negative.
    \end{itemize}
\end{itembox}

\section{LLCref}

\begin{theorem}
    \label{thm:sample}
    This is a sample theorem.
\end{theorem}

Use \cref{thm:sample} with cref instead of Theorem~\ref{thm:sample} with ref to avoid mistakes.

\section{LLDoubleQuotation}

Use ``XXX'' instead of “XXX” or "XXX".

\section{LLENDash}

\begin{itemize}
    \item Erdos-Renyi (random graph, Erd\H{o}s--R\'enyi)
    \item Einstein-Podolsky-Rosen (quantum physics, Einstein--Podolsky--Rosen)
    \item Fruchterman-Reingold (graph drawing, Fruchterman--Reingold)
    \item Gauss-Legendre (numerical integration, Gauss--Legendre)
    \item Gibbs-Helmholtz (thermodynamics, Gibbs--Helmholtz)
    \item Karush-Kuhn-Tucker (optimization, Karush--Kuhn--Tucker)
\end{itemize}

Exception: Fritz-John (optimization, name of a person)

False Positive: Wrong-Example

\section{LLEqnarray}

\begin{eqnarray}
    x & = & y \\
    a & = & b
\end{eqnarray}

\section{LLLlGg}

$n << m$ and $m >> n$ should be $n \ll m$ and $m \gg n$.

I like human $<<<$ cat $<<<<<<<<<<<<<<<<$ dog.

\section{LLRefEq}

To refer to the equation, use \eqref{eq:sample} instead of (\ref{eq:sample}) to avoid mistakes forgetting to add parentheses.

\section{LLSharp}

\begin{table}[h]
    \centering
    \begin{tabular}{lll}
        $\backslash\#$    & $\#A$      & \tA{ok} \\
        $\backslash$sharp & $\sharp A$ & \tD{ng}
    \end{tabular}
\end{table}

\section{LLNonASCII}

 { }!"#$%&'()*+,-./

日本語の文章は、upLaTeXでフツウに書けます。

(You can write Japanese sentences as usual with upLaTeX.)

\section{LLSI}

Example: 10 KB, $3.5\mathrm{MiB}$, \SI{500}{\giga\byte}.

\newcommand{\EB}{Some Awesome Command.This is not ExaByte.}

\EB.

This 1EB is one ExaByte.

\section{LLT}

\begin{equation*}
    X^T \quad X^\top \quad X^{\mathsf{T}}
\end{equation*}

\section{LLTitle}

\subsection{non title case words}

\subsubsection{
    This Is a Correct Title
}

\subparagraph{SubParagraph: Test With Ref~\ref{thm:sample}}

\section{LLUserDefined}

You can define your own rule, such as prohibiting the use of a f\^{}a.

\begin{equation*}
    f^a(X) \quad f^{\mathrm{a}}(X)
\end{equation*}

\end{document}
